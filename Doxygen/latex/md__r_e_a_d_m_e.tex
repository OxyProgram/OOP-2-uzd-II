Objektinio programavimo 2 užduotis Programos veikimo principas\+:

Vartotojas yra paprašomas įvesti studento vardą, pavardę. Vartotojas yra paprašomas įvesti studento namų darbų rezultatus. Jeigu visi rezultatai suvesti, reikia įvesti $\ast$ (žvaigždutę). Vartotojas yra prašomas įvesti studento egzamino rezultatą. Vartotojas yra prašomas įveskito kito studento duomenis. Jeigu visi studentai įvesti, reikia įvesti $\ast$ (žvaigždutę). Vartotojas yra prašomas pasirinkti, ar skaičiuoti vidurkį, ar medianą. Atitinkamai reikia įrašyti 1 arba 2. Vartotojui sugeneruojama lentelė su studento duomenimis. Programos įdiegimas\+:

Atidaryti terminalą atsisiųstų failų aplanke. Į komandinę eilutę įrašyti \char`\"{}make\char`\"{} ir bus sugeneruotas output.\+exe failas. Paleisti output.\+exe failą.

Nuo v0.\+4\+: Vartotojas yra prašomas sugeneruoti pasirinkto dydžio studentų sąrašą. Pagal sugeneruotą failą terminale yra parodomas laikas, kurį užtruko programa tą sąrašą apdorojant.

Versijos\+:

v0.\+1 -\/ Įvykdyti užduoties reikalavimai. v0.\+2 -\/ Pridėta galimybė nuskaityti duomenis iš txt failo, studentų vardų rikiavimas alfabetiškai. v0.\+3 -\/ Atnaujinta programos struktūra, sukurti papildomi .cpp ir .h failai. v0.\+4 -\/ Sistemos tikslas pakeistas į tam tikrų užduočių laiko matavimą ir testavimą. v0.\+5 -\/ Skirtingų tipų konteinerių testavimas atliekant tam tikras užduotis. v1.\+0 -\/ Naudojami 2 konteineriai, paspartintas programos veikimas. v1.\+1 -\/ pakeista programos struktūra. Iš student struct pereitą prie class. v1.\+2 Pridėtos \char`\"{}gerosios praktikos\char`\"{} funkcijos klasei Student. v1.\+5 -\/ Pridėta abstrakti klasė \mbox{\hyperlink{class_person}{Person}}, kurią paveldi klasė Student. 